\documentclass{report}
\usepackage{luatexja} % LuaTeXで日本語を使うためのパッケージ
\usepackage{luatexja-fontspec} % LuaTeX用の日本語フォント設定

% --- 数学関連 ---
\usepackage{amsmath, amssymb, amsfonts, mathtools, bm, amsthm} % 基本的な数学パッケージ
\usepackage{type1cm, upgreek} % 数式フォントとギリシャ文字
\usepackage{physics, mhchem} % 物理や化学の記号や式の表記を簡単にする

% --- 表関連 ---
\usepackage{multirow, longtable, tabularx, array, colortbl, dcolumn, diagbox} % 表のレイアウトを柔軟にする
\usepackage{tablefootnote, truthtable} % 表中に注釈を追加、真理値表
\usepackage{tabularray} % 高度な表組みレイアウト

% --- グラフィック関連 ---
\usepackage{tikz, graphicx} % 図の描画と画像の挿入
\usepackage{caption, subcaption} % 図や表のキャプション設定
\usepackage{float, here} % 図や表の位置指定

% --- レイアウトとページ設定 ---
\usepackage{fancyhdr} % ページヘッダー、フッター、余白の設定
\usepackage[top = 20truemm, bottom = 20truemm, left = 20truemm, right = 20truemm]{geometry}
\usepackage{fancybox, ascmac} % ボックスのデザイン

% --- 色とスタイル ---
\usepackage{xcolor, color, colortbl, tcolorbox} % 色とカラーボックス
\usepackage{listings, jvlisting} % コードの色付けとフォーマット

% --- 参考文献関連 ---
\usepackage{biblatex, usebib} % 参考文献の管理と挿入
\usepackage{url, hyperref} % URLとリンクの設定

% --- その他の便利なパッケージ ---
\usepackage{footmisc} % 脚注のカスタマイズ
\usepackage{multicol} % 複数段組
\usepackage{comment} % コメントアウトの拡張
\usepackage{siunitx} % 単位の表記
\usepackage{docmute}
% \usepackage{appendix}
% --- tcolorboxとtikzの設定 ---
\tcbuselibrary{theorems, breakable} % 定理のボックスと改ページ設定
\usetikzlibrary{decorations.markings, arrows.meta, calc} % tikzの装飾や矢印の設定

% --- 定理スタイルと数式設定 ---
\theoremstyle{definition} % 定義スタイル
\numberwithin{equation}{section} % 式番号をサブセクション単位でリセット

% --- hyperrefの設定 ---
\hypersetup{
  setpagesize = false,
  bookmarks = true,
  bookmarksdepth = tocdepth,
  bookmarksnumbered = true,
  colorlinks = false,
  pdftitle = {}, % PDFタイトル
  pdfsubject = {}, % PDFサブジェクト
  pdfauthor = {}, % PDF作者
  pdfkeywords = {} % PDFキーワード
}

% --- siunitxの設定 ---
\sisetup{
  table-format = 1.5, % 小数点以下の桁数
  table-number-alignment = center, % 数値の中央揃え
}

% --- その他の設定 ---
\allowdisplaybreaks % 数式の途中改ページ許可
\newcolumntype{t}{!{\vrule width 0.1pt}} % 新しいカラムタイプ
\newcolumntype{b}{!{\vrule width 1.5pt}} % 太いカラム
\UseTblrLibrary{amsmath, booktabs, counter, diagbox, functional, hook, html, nameref, siunitx, varwidth, zref} % tabularrayのライブラリ
\setlength{\columnseprule}{0.4pt} % カラム区切り線の太さ
\captionsetup[figure]{font = bf} % 図のキャプションの太字設定
\captionsetup[table]{font = bf} % 表のキャプションの太字設定
\captionsetup[lstlisting]{font = bf} % コードのキャプションの太字設定
\captionsetup[subfigure]{font = bf, labelformat = simple} % サブ図のキャプション設定
\setcounter{secnumdepth}{4} % セクションの深さ設定
\newcolumntype{d}{D{.}{.}{5}} % 数値のカラム
\newcolumntype{M}[1]{>{\centering\arraybackslash}m{#1}} % センター揃えのカラム
\DeclareMathOperator{\diag}{diag}
\everymath{\displaystyle} % 数式のスタイル
\newcommand{\inner}[2]{\left\langle #1, #2 \right\rangle}
\renewcommand{\figurename}{図}
\renewcommand{\i}{\mathrm{i}} % 複素数単位i
\renewcommand{\laplacian}{\grad^2} % ラプラシアンの記号
\renewcommand{\thesubfigure}{(\alph{subfigure})} % サブ図の番号形式
\newcommand{\m}[3]{\multicolumn{#1}{#2}{#3}} % マルチカラムのショートカット
\renewcommand{\r}[1]{\mathrm{#1}} % mathrmのショートカット
\newcommand{\e}{\mathrm{e}} % 自然対数の底e
\newcommand{\Ef}{E_{\mathrm{F}}} % フェルミエネルギー
\renewcommand{\c}{\si{\degreeCelsius}} % 摂氏記号
\renewcommand{\d}{\r{d}} % d記号
\renewcommand{\t}[1]{\texttt{#1}} % タイプライタフォント
\newcommand{\kb}{k_{\mathrm{B}}} % ボルツマン定数
% \renewcommand{\phi}{\varphi} % ϕをφに変更
\renewcommand{\epsilon}{\varepsilon}
\newcommand{\fullref}[1]{\textbf{\ref{#1} \nameref{#1}}}
\newcommand{\reff}[1]{\textbf{図\ref{#1}}} % 図参照のショートカット
\newcommand{\reft}[1]{\textbf{表\ref{#1}}} % 表参照のショートカット
\newcommand{\refe}[1]{\textbf{式\eqref{#1}}} % 式参照のショートカット
\newcommand{\refp}[1]{\textbf{コード\ref{#1}}} % コード参照のショートカット
\renewcommand{\lstlistingname}{コード} % コードリストの名前
\renewcommand{\theequation}{\thesection.\arabic{equation}} % 式番号の形式
\renewcommand{\footrulewidth}{0.4pt} % フッターの線
\newcommand{\mar}[1]{\textcircled{\scriptsize #1}} % 丸囲み文字
\newcommand{\combination}[2]{{}_{#1} \mathrm{C}_{#2}} % 組み合わせ
\newcommand{\thline}{\noalign{\hrule height 0.1pt}} % 細い横線
\newcommand{\bhline}{\noalign{\hrule height 1.5pt}} % 太い横線

% --- カスタム色定義 ---
\definecolor{burgundy}{rgb}{0.5, 0.0, 0.13} % バーガンディ色
\definecolor{charcoal}{rgb}{0.21, 0.27, 0.31} % チャコール色
\definecolor{forest}{rgb}{0.0, 0.35, 0} % 森の緑色

% --- カスタム定理環境の定義 ---
\newtcbtheorem[number within = chapter]{myexc}{練習問題}{
  fonttitle = \gtfamily\sffamily\bfseries\upshape,
  colframe = forest,
  colback = forest!2!white,
  rightrule = 1pt,
  leftrule = 1pt,
  bottomrule = 2pt,
  colbacktitle = forest,
  theorem style = standard,
  breakable,
  arc = 0pt,
}{exc-ref}
\newtcbtheorem[number within = chapter]{myprop}{命題}{
  fonttitle = \gtfamily\sffamily\bfseries\upshape,
  colframe = blue!50!black,
  colback = blue!50!black!2!white,
  rightrule = 1pt,
  leftrule = 1pt,
  bottomrule = 2pt,
  colbacktitle = blue!50!black,
  theorem style = standard,
  breakable,
  arc = 0pt
}{proposition-ref}
\newtcbtheorem[number within = chapter]{myrem}{注意}{
  fonttitle = \gtfamily\sffamily\bfseries\upshape,
  colframe = yellow!20!black,
  colback = yellow!50,
  rightrule = 1pt,
  leftrule = 1pt,
  bottomrule = 2pt,
  colbacktitle = yellow!20!black,
  theorem style = standard,
  breakable,
  arc = 0pt
}{remark-ref}
\newtcbtheorem[number within = chapter]{myex}{例題}{
  fonttitle = \gtfamily\sffamily\bfseries\upshape,
  colframe = black,
  colback = white,
  rightrule = 1pt,
  leftrule = 1pt,
  bottomrule = 2pt,
  colbacktitle = black,
  theorem style = standard,
  breakable,
  arc = 0pt
}{example-ref}
\newtcbtheorem[number within = chapter]{exc}{Requirement}{myexc}{exc-ref}
\newcommand{\rqref}[1]{{\bfseries\sffamily 練習問題 \ref{exc-ref:#1}}}
\newtcbtheorem[number within = chapter]{definition}{Definition}{mydef}{definition-ref}
\newcommand{\dfref}[1]{{\bfseries\sffamily 定義 \ref{definition-ref:#1}}}
\newtcbtheorem[number within = chapter]{prop}{命題}{myprop}{proposition-ref}
\newcommand{\prref}[1]{{\bfseries\sffamily 命題 \ref{proposition-ref:#1}}}
\newtcbtheorem[number within = chapter]{rem}{注意}{myrem}{remark-ref}
\newcommand{\rmref}[1]{{\bfseries\sffamily 注意 \ref{remark-ref:#1}}}
\newtcbtheorem[number within = chapter]{ex}{例題}{myex}{example-ref}
\newcommand{\exref}[1]{{\bfseries\sffamily 例題 \ref{example-ref:#1}}}

\pagestyle{fancy} % ヘッダー・フッターのスタイル設定
\fancyhead[R]{\rightmark}
\renewcommand{\subsectionmark}[1]{\markright{\thesubsection\ #1}}
\cfoot{\thepage} % 中央フッターにページ番号
\lhead{}
\rfoot{Yuto Masuda and Haruki Aoki} % 右フッターに名前
\setcounter{tocdepth}{4} % 目次の深さ
\makeatletter
\@addtoreset{equation}{section} % セクションごとに式番号をリセット
\makeatother

\title{場の量子論ノート}
\date{更新日: \today}
\author{Haruki AOKI}


\begin{document}
  本章では,これから議論する場の量子論の準備を行う.
  断りの無い限り,$c = \hbar = 1$なる自然単位系を用いる.
  場の量子論は,既存の量子力学などの物理法則あるいは方程式を\textbf{Poincar\'e変換}に対して不変な形に書き直す理論である.
  ただし,Poincar\'e変換は,Lorentz変換と時空並進変換のことである.
  \section{記法}
    まず,微小な時空間上の2点,$\qty(t, x, y, z)$,$\qty(t + \dd{t}, x + \dd{x}, y + \dd{y}, z + \dd{z})$に対して,世界長さ$\dd{s^2}$を考える.
    \begin{align}
      \dd{s^2} \coloneqq \dd{t}^2 - \qty(\dd{x}^2 + \dd{y}^2 + \dd{z}^2)
    \end{align}
    4次元時空座標を,
    \begin{align}
      x^{\mu} &\coloneqq \qty(t, x, y, z) \\ 
      x_{\nu} &\coloneqq \qty(t, -x, -y, -z)
    \end{align}
    と定義する.
    Einsteinの縮約を使っていることに注意する.
    計量テンソル$\eta_{\mu\nu}$,$\eta^{\mu\nu}$を,
    \begin{align}
      x_{\nu} &= \eta_{\mu\nu}x^{\mu} \\ 
      x^{\nu} &= x_{\mu}\eta^{\mu\nu}
    \end{align}
    となるように定義する.
    計量テンソルを用いれば,
    \begin{align}
      \dd{s^2} &= \eta_{\mu\nu}\dd{x^{\mu}}\dd{x^{\nu}} \label{def-of-proper-length} \\ 
      &= \dd{x_{\mu}}\dd{x_{\nu}}\eta^{\mu\nu}
    \end{align}
    と書ける.
    $\eta_{\mu\nu}$は,上付き添え字が$k$個,下付き添え字が$l$個あるものに対して,添え字を上付き添え字を$k + l - 2$個,下付き添え字を$k + l + 2$個にするものだと考えてよい.
    なお,負の添え字の数は,添え字の上下を逆転させたものと考える.
    同様に,$\eta^{\mu\nu}$は,上付き添え字が$k$個,下付き添え字が$l$個あるものに対して,添え字を上付き添え字を$k + l + 2$個,下付き添え字を$k + l - 2$個にするものだと考えてよい.
    \par
    次に,全微分は,
    \begin{align}
      \partial_{\nu} &\coloneqq \pdv{x^{\nu}} \\ 
      \partial^{\nu} &\coloneqq \pdv{x_{\nu}} 
    \end{align}
    と定義される.
  \section{Poincar\'e変換}
    Poincar\'e変換は,Lorentz変換のパラメータを$\Lambda_{\nu}^{\mu}$,時空並進のパラメータを$a^{\mu}$とすると,
    \begin{align}
      x'^{\mu} = \Lambda_{\nu}^{\mu}x^{\nu} + a^{\mu}
    \end{align}
    と書ける.
    微小変位は,
    \begin{align}
      \dd{x'^{\mu}} &= \dd{\qty(\Lambda_{\nu}^{\mu}x^{\nu} + a^{\mu})} \\ 
      &= \Lambda_{\nu}^{\mu}\dd{x^{\nu}}
    \end{align}
    と書けるから,世界長さ$\dd{s^2}$は,
    \begin{align}
      \dd{s'^2} &= \eta_{\rho\lambda}\dd{x'^{\rho}}\dd{x'^{\lambda}} \\ 
      &= \eta_{\rho\lambda}\qty(\Lambda_{\nu}^{\rho}\dd{x^{\nu}})\qty(\Lambda_{\mu}^{\lambda}\dd{x^{\mu}}) \\ 
      &= \eta_{\rho\lambda}\Lambda_{\nu}^{\rho}\Lambda_{\mu}^{\lambda}\dd{x^{\nu}}\dd{x^{\mu}}
    \end{align}
    となる.
    今,Poincar\'e変換に対して方程式は不変であることが要請されているのであった.
    \refe{def-of-proper-length}で与えられる世界長さを与える方程式もPoincar\'e変換に対して不変であるべきだから,
    \begin{align}
      \dd{s'^2} &= \dd{s^2} \\ 
      \Leftrightarrow \eta_{\rho\lambda}\Lambda_{\nu}^{\rho}\Lambda_{\mu}^{\lambda}\dd{x^{\nu}}\dd{x^{\mu}} &= \eta_{\mu\nu}\dd{x^{\mu}}\dd{x^{\nu}} \\ 
      \Leftrightarrow \eta_{\rho\lambda}\Lambda_{\nu}^{\rho}\Lambda_{\mu}^{\lambda} &= \eta_{\mu\nu} \\ 
      \Leftrightarrow \eta_{\rho\lambda}\Lambda_{\mu}^{\lambda} &= \eta_{\mu\nu}\qty(\Lambda^{-1})_{\rho}^{\nu}\label{eta-lambda} \\ 
      \Leftrightarrow \eta_{\rho\lambda} &= \eta_{\mu\nu}\qty(\Lambda^{-1})_{\rho}^{\nu}\qty(\Lambda^{-1})_{\lambda}^{\mu}
    \end{align}
    である.
    ただし,
    \begin{align}
      \Lambda_{\nu}^{\rho}\qty(\Lambda^{-1})_{\rho}^{\nu} = \qty(\Lambda^{-1})_{\rho}^{\nu}\Lambda_{\nu}^{\rho} = 1
    \end{align}
    なる関係を用いた.
    まとめると,$x^{\mu}$と$x_{\mu}$の変換性は\reff{transform}のようになる.
    \begin{figure}[H]
      \centering
      \begin{tikzpicture}[scale = 2]
        \node (a) at (0, 1.2) {$x^{\mu}$};
        \node (x) at (1.2, 1.2) {$x_{\mu}$};
        \node (b) at (0, 0) {$x'^{\mu}$};
        \node (y) at (1.2, 0) {$x'_{\mu}$};
        \draw[above, ->] (a) to node {$\scriptstyle \eta$} (x);
        \draw[right, ->] (x) to node {$\scriptstyle \Lambda^{-1}$} (y);
        \draw[left, ->] (a) to node[swap] {$\scriptstyle \Lambda$} (b);
        \draw[below, ->] (b) to node[swap] {$\scriptstyle \eta$} (y);
        \end{tikzpicture}
      \caption{$x^{\mu}$と$x_{\mu}$の変換性}\label{transform}
    \end{figure}
    $x^{\mu}$から$x'_{\mu}$への変換がwell-definedであることは非自明なので確かめておこう.
    まず,$x^{\mu} \to x_{\mu} \to x'_{\mu}$のとき,
    \begin{align}
      x_{\mu} &= \eta_{\mu\nu}x^{\nu} \\ 
      x'_{\mu} &= x_{\lambda}\qty(\Lambda^{-1})_{\mu}^{\lambda}
    \end{align}
    次に,$x^{\mu} \to x'^{\mu} \to x'_{\mu}$のとき,
    \refe{eta-lambda}より,
    \begin{align}
      \eta_{\rho\lambda}\Lambda_{\nu}^{\rho} = \eta_{\rho\lambda}\qty(\Lambda^{-1})_{\mu}^{\lambda}
    \end{align}
    であることを用いると,
    \begin{align}
      x'^{\mu} &= \Lambda_{\rho}^{\mu}x^{\rho} \\ 
      x'_{\mu} &= \eta_{\mu\nu}x'^{\nu} \\ 
      &= \eta_{\mu\nu}\Lambda_{\rho}^{\nu}x^{\rho} \\ 
      &= \eta_{\rho\lambda}x^{\rho}\qty(\Lambda^{-1})_{\mu}^{\lambda} \\ 
      &= x_{\lambda}\qty(\Lambda^{-1})_{\mu}^{\lambda}
    \end{align}
    となり,$\eta$と$\Lambda$による変換はwell-definedであることが分かる.
    \par
    また,全微分について,
    \begin{align}
      \partial'^{\nu} &\coloneqq \pdv{x'_{\nu}} \\ 
      &= \pdv{x_{\rho}}{x'_{\nu}}\pdv{x_{\rho}} \\ 
      &= \qty[\qty(\Lambda^{-1})_{\rho}^{\nu}]^{-1}\partial_{\rho} \\ 
      &= \Lambda_{\rho}^{\nu}\partial^{\rho} \\ 
      \partial'_{\nu} &\coloneqq \pdv{x'^{\nu}} \\ 
      &= \pdv{x^{\rho}}{x'^{\nu}}\pdv{x^{\rho}} \\ 
      &= \qty(\Lambda_{\rho}^{\nu})^{-1}\partial_{\rho} \\ 
      &= \partial_{\rho}\qty(\Lambda^{-1})_{\nu}^{\rho}
    \end{align}
    なる関係がある.Lorentz変換をPoincar\'e変換に変更しても定数の微分は0なので,同様である.
  \section{スカラー・ベクトル・テンソル}
    本節ではスカラー・ベクトル・テンソルを定義する.
    Lorentz変換のパラメータを$\Lambda$とする.
    \subsection{スカラー}
      スカラーはLorentz変換に対して不変な量である.
      すなわち,
      \begin{align}
        S \mapsto S \eqqcolon S'
      \end{align}
      なる量である.
    \subsection{ベクトル}
      ベクトルは2種類あり,Lorentz変換によって時空座標を変換したときに,時空座標$x^{\mu}$と同じように変換される反変ベクトルと,
      $x_{\mu}$と同じように変換される共変ベクトルに分けられる.
      すなわち,
      \begin{align}
        A^{\mu} &\mapsto \Lambda_{\nu}^{\mu}A^{\nu} \eqqcolon A'^{\mu} \\ 
        B_{\mu} &\mapsto B_{\nu}\qty(\Lambda^{-1})_{\mu}^{\nu} \eqqcolon B'_{\mu}
      \end{align}
      において,$A^{\mu}$,$A'^{\mu}$が反変ベクトル,$B_{\mu}$,$B'_{\mu}$が共変ベクトルである.
    \subsection{テンソル}
      テンソルは3種類あり,Lorentz変換によって時空座標を変換したときに,時空座標$x^{\mu}$を2回変換したとき同じように変換される2階の反変テンソル,
      時空座標$x^{\mu}$を1回変換してから1回逆変換したとき同じように変換される2階の混合テンソル,
      時空座標$x_{\mu}$を2回変換したとき同じように変換される2階の共変テンソルの3つに分けられる.
      すなわち,
      \begin{align}
        T^{\mu\nu} &\mapsto \Lambda_{\rho}^{\mu}\Lambda_{\lambda}^{\nu}T^{\rho\lambda} \eqqcolon T'^{\mu\nu} \\ 
        T_{\nu}^{\mu} &\mapsto \Lambda_{\rho}^{\mu}T_{\lambda}^{\rho}\qty(\Lambda^{-1})_{\nu}^{\lambda} \eqqcolon {T'}_{\nu}^{\mu} \\ 
        T_{\nu\mu} &\mapsto T_{\lambda}^{\rho}\qty(\Lambda^{-1})_{\mu}^{\rho}\qty(\Lambda^{-1})_{\nu}^{\lambda} \eqqcolon T'_{\nu\mu}
      \end{align}
      の3つがある.
      3つのテンソルの間には,
      \begin{align}
        T^{\mu\nu} = T_{\lambda}^{\mu}\eta^{\nu\lambda} = T_{\rho\lambda}\eta^{\mu\rho}\eta^{\nu\lambda}
      \end{align}
      なる関係がある.
\end{document}