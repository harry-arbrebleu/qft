\documentclass{report}
\input{../head.tex}
\begin{document}
  \section{Klein-Gordon方程式の「導出」}
    非相対論的・古典的エネルギーの関係式,
    \begin{align}
      E = \frac{\bm{p}^2}{2m}
    \end{align}
    の両辺に波動函数$\psi(t, \bm{x})$掛けて
    \begin{align}
      E &\to \i\pdv{t} \label{energy-to-diff}\\ 
      \bm{p} &\to -\i\grad \label{momentum-to-diff}
    \end{align}
    なる変換を行えば,
    \begin{align}
      E &= \frac{\bm{p}^2}{2m} \\ 
      \Rightarrow E\psi(t, \bm{x}) &= \frac{\bm{p}^2}{2m}\psi(t, \bm{x}) \\ 
      \Rightarrow \i\pdv{t}\psi(t, \bm{x}) &= \frac{1}{2m}\qty(-\i\grad)^2\psi(t, \bm{x}) \\ 
      \Rightarrow \i\pdv{t}\psi(t, \bm{x}) &= -\frac{1}{2m}\laplacian\psi
    \end{align}
    となり,Schr\"odinger方程式を得る.
    \par
    では相対論的なエネルギーの関係式,
    \begin{align}
      E^2 = \bm{p}^2 + m^2\label{relativity-energy}
    \end{align}
    を変換すると,どのようになるだろう.
    自然単位系を用いているため,静止エネルギーの2乗について$m^2c^4 = m^2$となっていることに注意する.
    \refe{relativity-energy}の両辺に波動函数$\psi(x^{\mu})$掛けて,
    \refe{energy-to-diff},\refe{momentum-to-diff}を用いれば,
    \begin{align}
      E^2 &= \bm{p}^2 + m^2 \\ 
      E\psi(x^{\mu}) &= \frac{\bm{p}^2}{2m}\psi(x^{\mu}) + m^2\psi(x^{\mu}) \\ 
      \qty(\i\pdv{t})^2\psi(x^{\mu}) &= \qty[\qty(-\i\grad)^2 + m^2]\psi(x^{\mu}) \\ 
      -\pdv[2]{t}\psi(x^{\mu}) &= -\laplacian\psi(x^{\mu}) + m^2\psi(x^{\mu}) \\ 
      \qty(\partial_{\mu}\partial^{\mu} - m^2)\psi(x^{\mu}) &= 0
    \end{align}
    が成立する.
    \par
    Klein-Gordon方程式がPoincar\'e変換に対して不変であるためには,
    \begin{align}
      \qty(\partial'_{\nu}\partial'^{\nu} - m^2)\psi'(x^{\mu}) &= 0 \\ 
      \Leftrightarrow \qty(\partial_{\rho}\qty(\Lambda^{-1})_{\nu}^{\rho}\Lambda_{\rho}^{\nu}\partial^{\rho} - m^2)\psi'(x'^{\mu}) &= 0 \\ 
      \Leftrightarrow \qty(\partial_{\rho}\partial^{\rho} - m^2)\psi'(x'^{\mu}) &= 0
    \end{align}
    であることより,
    \begin{align}
      \psi'\qty(x'^{\mu}) = \psi(x^{\mu})
    \end{align}
    が成立すればよい.
    このような函数$\psi(x)$のことをスカラー函数という.
\end{document}