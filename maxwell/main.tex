\documentclass{report}
\usepackage{luatexja} % LuaTeXで日本語を使うためのパッケージ
\usepackage{luatexja-fontspec} % LuaTeX用の日本語フォント設定

% --- 数学関連 ---
\usepackage{amsmath, amssymb, amsfonts, mathtools, bm, amsthm} % 基本的な数学パッケージ
\usepackage{type1cm, upgreek} % 数式フォントとギリシャ文字
\usepackage{physics, mhchem} % 物理や化学の記号や式の表記を簡単にする

% --- 表関連 ---
\usepackage{multirow, longtable, tabularx, array, colortbl, dcolumn, diagbox} % 表のレイアウトを柔軟にする
\usepackage{tablefootnote, truthtable} % 表中に注釈を追加、真理値表
\usepackage{tabularray} % 高度な表組みレイアウト

% --- グラフィック関連 ---
\usepackage{tikz, graphicx} % 図の描画と画像の挿入
\usepackage{caption, subcaption} % 図や表のキャプション設定
\usepackage{float, here} % 図や表の位置指定

% --- レイアウトとページ設定 ---
\usepackage{fancyhdr} % ページヘッダー、フッター、余白の設定
\usepackage[top = 20truemm, bottom = 20truemm, left = 20truemm, right = 20truemm]{geometry}
\usepackage{fancybox, ascmac} % ボックスのデザイン

% --- 色とスタイル ---
\usepackage{xcolor, color, colortbl, tcolorbox} % 色とカラーボックス
\usepackage{listings, jvlisting} % コードの色付けとフォーマット

% --- 参考文献関連 ---
\usepackage{biblatex, usebib} % 参考文献の管理と挿入
\usepackage{url, hyperref} % URLとリンクの設定

% --- その他の便利なパッケージ ---
\usepackage{footmisc} % 脚注のカスタマイズ
\usepackage{multicol} % 複数段組
\usepackage{comment} % コメントアウトの拡張
\usepackage{siunitx} % 単位の表記
\usepackage{docmute}
% \usepackage{appendix}
% --- tcolorboxとtikzの設定 ---
\tcbuselibrary{theorems, breakable} % 定理のボックスと改ページ設定
\usetikzlibrary{decorations.markings, arrows.meta, calc} % tikzの装飾や矢印の設定

% --- 定理スタイルと数式設定 ---
\theoremstyle{definition} % 定義スタイル
\numberwithin{equation}{section} % 式番号をサブセクション単位でリセット

% --- hyperrefの設定 ---
\hypersetup{
  setpagesize = false,
  bookmarks = true,
  bookmarksdepth = tocdepth,
  bookmarksnumbered = true,
  colorlinks = false,
  pdftitle = {}, % PDFタイトル
  pdfsubject = {}, % PDFサブジェクト
  pdfauthor = {}, % PDF作者
  pdfkeywords = {} % PDFキーワード
}

% --- siunitxの設定 ---
\sisetup{
  table-format = 1.5, % 小数点以下の桁数
  table-number-alignment = center, % 数値の中央揃え
}

% --- その他の設定 ---
\allowdisplaybreaks % 数式の途中改ページ許可
\newcolumntype{t}{!{\vrule width 0.1pt}} % 新しいカラムタイプ
\newcolumntype{b}{!{\vrule width 1.5pt}} % 太いカラム
\UseTblrLibrary{amsmath, booktabs, counter, diagbox, functional, hook, html, nameref, siunitx, varwidth, zref} % tabularrayのライブラリ
\setlength{\columnseprule}{0.4pt} % カラム区切り線の太さ
\captionsetup[figure]{font = bf} % 図のキャプションの太字設定
\captionsetup[table]{font = bf} % 表のキャプションの太字設定
\captionsetup[lstlisting]{font = bf} % コードのキャプションの太字設定
\captionsetup[subfigure]{font = bf, labelformat = simple} % サブ図のキャプション設定
\setcounter{secnumdepth}{4} % セクションの深さ設定
\newcolumntype{d}{D{.}{.}{5}} % 数値のカラム
\newcolumntype{M}[1]{>{\centering\arraybackslash}m{#1}} % センター揃えのカラム
\DeclareMathOperator{\diag}{diag}
\everymath{\displaystyle} % 数式のスタイル
\newcommand{\inner}[2]{\left\langle #1, #2 \right\rangle}
\renewcommand{\figurename}{図}
\renewcommand{\i}{\mathrm{i}} % 複素数単位i
\renewcommand{\laplacian}{\grad^2} % ラプラシアンの記号
\renewcommand{\thesubfigure}{(\alph{subfigure})} % サブ図の番号形式
\newcommand{\m}[3]{\multicolumn{#1}{#2}{#3}} % マルチカラムのショートカット
\renewcommand{\r}[1]{\mathrm{#1}} % mathrmのショートカット
\newcommand{\e}{\mathrm{e}} % 自然対数の底e
\newcommand{\Ef}{E_{\mathrm{F}}} % フェルミエネルギー
\renewcommand{\c}{\si{\degreeCelsius}} % 摂氏記号
\renewcommand{\d}{\r{d}} % d記号
\renewcommand{\t}[1]{\texttt{#1}} % タイプライタフォント
\newcommand{\kb}{k_{\mathrm{B}}} % ボルツマン定数
% \renewcommand{\phi}{\varphi} % ϕをφに変更
\renewcommand{\epsilon}{\varepsilon}
\newcommand{\fullref}[1]{\textbf{\ref{#1} \nameref{#1}}}
\newcommand{\reff}[1]{\textbf{図\ref{#1}}} % 図参照のショートカット
\newcommand{\reft}[1]{\textbf{表\ref{#1}}} % 表参照のショートカット
\newcommand{\refe}[1]{\textbf{式\eqref{#1}}} % 式参照のショートカット
\newcommand{\refp}[1]{\textbf{コード\ref{#1}}} % コード参照のショートカット
\renewcommand{\lstlistingname}{コード} % コードリストの名前
\renewcommand{\theequation}{\thesection.\arabic{equation}} % 式番号の形式
\renewcommand{\footrulewidth}{0.4pt} % フッターの線
\newcommand{\mar}[1]{\textcircled{\scriptsize #1}} % 丸囲み文字
\newcommand{\combination}[2]{{}_{#1} \mathrm{C}_{#2}} % 組み合わせ
\newcommand{\thline}{\noalign{\hrule height 0.1pt}} % 細い横線
\newcommand{\bhline}{\noalign{\hrule height 1.5pt}} % 太い横線

% --- カスタム色定義 ---
\definecolor{burgundy}{rgb}{0.5, 0.0, 0.13} % バーガンディ色
\definecolor{charcoal}{rgb}{0.21, 0.27, 0.31} % チャコール色
\definecolor{forest}{rgb}{0.0, 0.35, 0} % 森の緑色

% --- カスタム定理環境の定義 ---
\newtcbtheorem[number within = chapter]{myexc}{練習問題}{
  fonttitle = \gtfamily\sffamily\bfseries\upshape,
  colframe = forest,
  colback = forest!2!white,
  rightrule = 1pt,
  leftrule = 1pt,
  bottomrule = 2pt,
  colbacktitle = forest,
  theorem style = standard,
  breakable,
  arc = 0pt,
}{exc-ref}
\newtcbtheorem[number within = chapter]{myprop}{命題}{
  fonttitle = \gtfamily\sffamily\bfseries\upshape,
  colframe = blue!50!black,
  colback = blue!50!black!2!white,
  rightrule = 1pt,
  leftrule = 1pt,
  bottomrule = 2pt,
  colbacktitle = blue!50!black,
  theorem style = standard,
  breakable,
  arc = 0pt
}{proposition-ref}
\newtcbtheorem[number within = chapter]{myrem}{注意}{
  fonttitle = \gtfamily\sffamily\bfseries\upshape,
  colframe = yellow!20!black,
  colback = yellow!50,
  rightrule = 1pt,
  leftrule = 1pt,
  bottomrule = 2pt,
  colbacktitle = yellow!20!black,
  theorem style = standard,
  breakable,
  arc = 0pt
}{remark-ref}
\newtcbtheorem[number within = chapter]{myex}{例題}{
  fonttitle = \gtfamily\sffamily\bfseries\upshape,
  colframe = black,
  colback = white,
  rightrule = 1pt,
  leftrule = 1pt,
  bottomrule = 2pt,
  colbacktitle = black,
  theorem style = standard,
  breakable,
  arc = 0pt
}{example-ref}
\newtcbtheorem[number within = chapter]{exc}{Requirement}{myexc}{exc-ref}
\newcommand{\rqref}[1]{{\bfseries\sffamily 練習問題 \ref{exc-ref:#1}}}
\newtcbtheorem[number within = chapter]{definition}{Definition}{mydef}{definition-ref}
\newcommand{\dfref}[1]{{\bfseries\sffamily 定義 \ref{definition-ref:#1}}}
\newtcbtheorem[number within = chapter]{prop}{命題}{myprop}{proposition-ref}
\newcommand{\prref}[1]{{\bfseries\sffamily 命題 \ref{proposition-ref:#1}}}
\newtcbtheorem[number within = chapter]{rem}{注意}{myrem}{remark-ref}
\newcommand{\rmref}[1]{{\bfseries\sffamily 注意 \ref{remark-ref:#1}}}
\newtcbtheorem[number within = chapter]{ex}{例題}{myex}{example-ref}
\newcommand{\exref}[1]{{\bfseries\sffamily 例題 \ref{example-ref:#1}}}

\pagestyle{fancy} % ヘッダー・フッターのスタイル設定
\fancyhead[R]{\rightmark}
\renewcommand{\subsectionmark}[1]{\markright{\thesubsection\ #1}}
\cfoot{\thepage} % 中央フッターにページ番号
\lhead{}
\rfoot{Yuto Masuda and Haruki Aoki} % 右フッターに名前
\setcounter{tocdepth}{4} % 目次の深さ
\makeatletter
\@addtoreset{equation}{section} % セクションごとに式番号をリセット
\makeatother

\title{場の量子論ノート}
\date{更新日: \today}
\author{Haruki AOKI}


\begin{document}
  \section{ゲージ場の導入}
    自然単位系でMaxwellの方程式を書くと,
    \begin{align}
      \div\bm{B} = 0 \label{maxwell-1} \\ 
      \curl\bm{E} + \pdv{\bm{B}}{t} = 0 \label{maxwell-2} \\ 
      \div\bm{E} = \rho \label{maxwell-3} \\ 
      \curl\bm{B} - \pdv{\bm{E}}{t} = \bm{j}\label{maxwell-4} 
    \end{align}
    となる.
    $\rho$は電荷密度,$\bm{j}$は電流密度である.
    \refe{maxwell-1}なる$\bm{B}$は,
    \begin{align}
      \bm{B} = \curl{\bm{A}}\label{b-gauge}
    \end{align}
    である.
    \refe{b-gauge}を\refe{maxwell-2}に代入すると,
    \begin{align}
      \curl\qty(\bm{E} + \pdv{\bm{A}}{t}) = 0\label{e-gauge-1}
    \end{align}
    となる.
    \refe{e-gauge-1}ようなベクトルは,
    \begin{align}
      &\bm{E} + \pdv{\bm{A}}{t} = -\grad A^0 \\ 
      \iff& \bm{E} = -\grad A^0 -\pdv{\bm{A}}{t} \label{e-gauge}
    \end{align}
    となる.
    \refe{b-gauge}と\refe{e-gauge}をまとめて書くと,
    \begin{align}
      \mqty(B^1 \\ B^2 \\ B^3) &= \mqty(\partial_2A^3 - \partial_3A^2 \\ \partial_3A^1 - \partial_1A^3 \\ \partial_1A^2 - \partial_2A^1) \\ 
      \mqty(E^1 \\ E^2 \\ E^3) &= -\mqty(\partial_0A^1 + \partial_1A^0  \\ \partial_0A^2 + \partial_2A^0  \\ \partial_0A^3 + \partial_3A^0 )
    \end{align}
    となる.
  \section{場の強さ}
    場の強さ$F^{\mu\nu}$なるテンソルを,
    \begin{align}
      F^{\mu\nu} \coloneqq \partial^{\mu}A^{\nu} - \partial^{\nu}A^{\mu}
    \end{align}
    と定義する.
    \par
    場の強さに関する性質をいくつか示す.
    まず,場の強さ$F^{\mu\nu}$は反対称テンソルであることを示す.
    \begin{proof}
      \begin{align}
        F^{\nu\mu} &= \partial^{\nu}A^{\mu} - \partial^{\mu}A^{\nu} \\ 
        &= -\partial^{\mu}A^{\nu} + \partial^{\nu}A^{\mu} \\ 
        &= -F^{\mu\nu}
      \end{align}
      であるから,$F^{\mu\nu}$は反対称テンソルである.
    \end{proof}
    電場と磁場は場の強さを用いて簡単に書くことが出来る.
    まず,磁場について$(\rho, \mu, \nu) \in \qty{(1, 2, 3), (2, 3, 1), (3, 1, 2)}$として,
    \begin{align}
      B^{\rho} &= \partial_{\mu}A^{\nu} - \partial_{\nu}A^{\mu} \\ 
      &= -\qty(\partial^{\mu}A^{\nu} - \partial^{\nu}A^{\mu}) \\ 
      &= -F^{\mu\nu} \label{b-field-strength}
    \end{align}
    と書ける.
    次に,電場について$\nu \in \qty{1, 2, 3}$として,
    \begin{align}
      E^{\nu} &= -\qty(\partial_0A^{\nu} + \partial_{\nu}A^0) \\ 
      &= -\qty(\partial^0A^{\nu} - \partial^{\nu}A^0) \\ 
      &= -F^{0\nu} \label{e-field-strength}
    \end{align}
    と書くことができる.
    \par
    $M^{\rho\mu\nu}$を,
    \begin{align}
      M^{\rho\mu\nu} \coloneqq \partial^{\rho}F^{\mu\nu} + \partial^{\mu}F^{\nu\rho} + \partial^{\nu}F^{\rho\mu}
    \end{align}
    と定義する.
    $M^{\rho\mu\nu}$が完全反対称テンソルであることを示す.
    \begin{proof}
      計算において,$F^{\mu\nu}$が反対称テンソルであることを用いる.
      3種類の添え字の入れ替えを考える.
      \begin{enumerate}
        \item~\\ 
          まず,$M^{\rho\mu\nu}$から1番目の添え字$\rho$と2番目の添え字$\mu$を入れ替えた量$M^{\mu\rho\nu}$を考えると,
          \begin{align}
            M^{\mu\rho\nu} &= \partial^{\mu}F^{\rho\nu} + \partial^{\rho}F^{\nu\mu} + \partial^{\nu}F^{\mu\rho} \\ 
            &= -\partial^{\mu}F^{\nu\rho} - \partial^{\rho}F^{\mu\nu} - \partial^{\nu}F^{\rho\mu} \\ 
            &= -\partial^{\rho}F^{\mu\nu} - \partial^{\mu}F^{\nu\rho} - \partial^{\nu}F^{\rho\mu} = -M^{\rho\mu\nu}
          \end{align}
          となる.
        \item~\\
          次に,$M^{\rho\mu\nu}$から2番目の添え字$\mu$と3番目の添え字$\nu$を入れ替えた量$M^{\rho\nu\mu}$を考えると,
          \begin{align}
            M^{\rho\nu\mu} &= \partial^{\rho}F^{\nu\mu} + \partial^{\nu}F^{\mu\rho} + \partial^{\mu}F^{\rho\nu} \\ 
            &= -\partial^{\rho}F^{\mu\nu} - \partial^{\nu}F^{\rho\mu} - \partial^{\mu}F^{\nu\rho} \\ 
            &= -\partial^{\rho}F^{\mu\nu} - \partial^{\mu}F^{\nu\rho} - \partial^{\nu}F^{\rho\mu} = -M^{\rho\mu\nu}
          \end{align}
          となる.
        \item~\\
          最後に,$M^{\rho\mu\nu}$から2番目の添え字$\mu$と3番目の添え字$\nu$を入れ替えた量$M^{\nu\mu\rho}$を考えると,
          \begin{align}
            M^{\nu\mu\rho} &= \partial^{\mu}F^{\nu\rho} + \partial^{\mu}F^{\rho\nu} + \partial^{\rho}F^{\nu\mu} \\ 
            &= -\partial^{\mu}F^{\rho\nu} - \partial^{\mu}F^{\nu\rho} + \partial^{\rho}F^{\mu\nu} \\ 
            &= -\partial^{\rho}F^{\mu\nu} - \partial^{\mu}F^{\nu\rho} - \partial^{\nu}F^{\rho\mu} = -M^{\rho\mu\nu}
          \end{align}
          となりる.
      \end{enumerate}
      以上より,$M^{\rho\mu\nu}$はどの2つの添え字を入れ替えても反対称であるから,完全反対称テンソルである.
    \end{proof}
    次に,場の強さを用いると\refe{maxwell-1}と\refe{maxwell-2}は,
    \begin{align}
      M^{\rho\mu\nu} = 0\label{maxwell-12-1}
    \end{align}
    と書けることを示す.
    \begin{proof}
      $M^{\rho\mu\nu}$は完全反対称テンソルであるので,いずれか2つの添え字が同じであれば$M^{\rho\mu\nu} = 0$であるので,
      \refe{maxwell-12-1}は自明に成立する.
      また,添え字の入れ替えで負号がつくので,$\rho < \mu < \nu$を課しても対称性を失わない.
      よって,
      \begin{align}
        \qty(\rho, \mu, \nu) \in \qty{(0, 1, 2), (0, 1, 3), (0, 2, 3), (1, 2, 3)}
      \end{align}
      のみを考えればよい.
      場の強さを\refe{b-field-strength}と\refe{e-field-strength}を用いて電磁場で書き直すと,
      \begin{enumerate}
        \item $\qty(\rho, \mu, \nu) = (0, 1, 2)$のとき.
          \begin{align}
            M^{012} &= \partial^0F^{12} + \partial^1F^{20} + \partial^2F^{01} \\ 
            &= -\partial^0B^3 + \partial^1E^2 - \partial^2E^1 \\ 
            &= -\partial_0B^3 -\qty(\partial_1E^2 - \partial_2E^1) \\ 
            &= -\qty[\pdv{B^3}{t} + \pdv{E^2}{x^1} - \pdv{E^1}{x^2}] \\ 
            &= 0
          \end{align}
          となる.
          なお,Maxwellの方程式の第2式である\refe{maxwell-2}の$z$成分が0となることを用いた.
        \item $\qty(\rho, \mu, \nu) = (0, 1, 3)$のとき.
          \begin{align}
            M^{013} &= \partial^0F^{13} + \partial^1F^{30} + \partial^3F^{01} \\ 
            &= \partial^0B^2 + \partial^1E^2 - \partial^3E^1 \\ 
            &= \partial_0B^2 + \qty(\partial_3E^1 - \partial_1E^3) \\ 
            &= \pdv{B^2}{t} + \pdv{E^1}{x^3} - \pdv{E^3}{x^1} \\ 
            &= 0
          \end{align}
          となる.
          なお,Maxwellの方程式の第2式である\refe{maxwell-2}の$y$成分が0となることを用いた.
        \item $\qty(\rho, \mu, \nu) = (0, 2, 3)$のとき.
          \begin{align}
            M^{023} &= \partial^0F^{23} + \partial^1F^{20} + \partial^2F^{01} \\ 
            &= -\partial^0B^1 + \partial^2E^3 - \partial^3E^2 \\ 
            &= -\partial_0B^1 - \qty(\partial_3E^2 - \partial_2E^3) \\ 
            &= -\qty[\pdv{B^1}{t} + \qty(\pdv{E^2}{x^3} - \pdv{E^3}{x^2})] \\ 
            &= 0
          \end{align}
          となる.
          なお,Maxwellの方程式の第2式である\refe{maxwell-2}の$x$成分が0となることを用いた.
        \item $\qty(\rho, \mu, \nu) = (1, 2, 3)$のとき.
          \begin{align}
            M^{123} &= \partial^1F^{23} + \partial^2F^{31} + \partial^3F^{12} \\ 
            &= -\partial^1B^1 - \partial^2E^2 - \partial^3E^3 \\ 
            &= 0
          \end{align}
          となる.
          なお,Maxwellの方程式の第1式である\refe{maxwell-1}が0となることを用いた.
      \end{enumerate}
      よって,\refe{maxwell-12-1}が成立する.
    \end{proof}
\end{document}